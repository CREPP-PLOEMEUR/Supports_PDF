\addPartText{Introduction aux réseaux}

\part{Les réseaux}


Un réseau est un ensemble de machines qui communiquent entre elles avec différents protocoles.\\

\subsection{Les adresses MAC}
\subsection{Les adresse IP}
\subsection{Adresse de diffusion}

Une adresse IP est unique sur un même réseau.

Sur un réseau local, les adresses sont choisies par un serveur DHCP.

Les adresses peuvent être :

\begin{items}{blue}{\Bullet}
	\item Statiques : Un appareil sur le réseau possède la même adresse après connexion puis déconnexion.
	\item Dynamiques : l'adresse IP varie dans le temps au bout d'une période d'expiration après une déconnexion (bail d'une journée par exemple)
\end{items}


\begin{items}{blue}{\Bullet}
\item UDP : il n'y a pas de vérification d'arrivée des données.\\ Les paquets sont envoyés mais ils peuvent être perdus sans que les utilisateurs le sachent.
\item TCP : Lors de l'envoi de paquets, il y a un acquittement lors de l'envoi et de la reception.
\end{items}


\section{Récupération des adresse IP}

Il est possible de scanner le réseau sur lequel est votre machine.
Pour cela, on peut utiliser l'utilitaire \lbl{blue}{LIB}{nmap}\footnote{Pour installer la commande sous linux, vous pouvez saisir la commande \lbl{red}{title}{sudo apt-get install nmap}}


Une fois que vous avez obtenu votre masque de sous-réseau (par exemple \lbl{green}{IP}{192.168.0.0}), vous pouvz saisir la commande suivante : 


\begin{Bash}{Scan réseau}
sudo nmap -sP 192.168.0.1-255
\end{Bash}

Cette commande va donc scanner les machines ayant une adresse IP comprise entre \lbl{green}{IP}{192.168.0.1} et \lbl{green}{IP}{192.168.0.255}
