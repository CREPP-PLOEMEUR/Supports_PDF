\addPartText{Théorie sur les servo-moteur et applications pratiques avec Arduino}
\partImg{Les servomoteur}{\rootImages/img.jpg}{0.4}
\chapter{Présentation}

\newcommand{\servo}{servo-moteurs~}

Les \servo sont utilisés lorsqu'on souhaite un asservissement en position d'un axe de rotation\footnote{Pulse Width Modulation : Modulation par Largeur d'Impulsion}


\subsection{Asservissement}

Un asservissement est un processus de correction pour maintenir une consigne. \\
Par exemple, un régulateur de vitesse dans une voiture est un système asservi car la vitesse doit être constante quelle que soit la pente.


\subsection{Architecture}

\img{Images/servo/servo_input.png}{Constitution d'un servo-moteur}{1.6}

\subsection{Domaines d'application}

\begin{items}{blue}{\Bullet}
    \item Modélisme
    \item Robotique
\end{items}

\subsection{Commande des \servo}

Les \servo ont besoins d'être controlés via un signal PWM.

\section{Principe de la PWM}

La PWM est la création d'un signal numérique dont le temps à l'état haut est variable.\\
On fait varier le rapport cyclique (appelé $r$) qui est compris entre 0 et 1.

$$ r = \frac{T_{on}}{T_{signal}} $$




\begin{figure}[h]  
    \centering 
      \begin{subfigure}[b]{0.4\linewidth}
        \begin{graphic}{0.8}{1}{-1}{60}{-1}{7}{t(ms)}{Tension V}{}
            \addPoints{blue}{(0,5)(5,5)(5,0)(20,0)(20,5)(25,5)(25,0)(40,0)(40,5)(45,5)(45,0)(70,0)}
            \addPoints{green}{(0,1.25)(70, 1.25)}
          \addLegend{Tension PWM,~Tension moyenne}
          \end{graphic}%NO END  LINE HERE
        \caption{r=0.25} 
      \end{subfigure}
    \begin{subfigure}[b]{0.4\linewidth}
        \begin{graphic}{0.8}{1}{-1}{60}{-1}{7}{t(ms)}{Tension V}{}
        \addPoints{blue}{(0,5)(10,5)(10,0)(20,0)(20,5)(30,5)(30,0)(40,0)(40,5)(50,5)(50,0)(70,0)}
        \addPoints{green}{(0,2.5)(70, 2.5)}
        \addLegend{Tension PWM,~Tension moyenne}
        \end{graphic}%NO END  LINE HERE
    \caption{r=0.5}
    \end{subfigure}
    \caption{Différents rapports cycliques}
\end{figure}  


\subsection{Applications de la PWM}


En faisant varier la tension de sortie dans le temps rapidement (>50Hz), on peut simuler une tension analogique.
Quelques applications: 

\begin{items}{blue}{\Bullet}
    \item Controle de la luminosté d'une LED
    \item Controle de \servo
\end{items}

\subsection{Code Arduino}

Voici un code d'exemple pour faire varier la luminosité d'une LED.

\begin{Cpp}{Variation de la luminosité d'une LED}

  const int pin_led = 11; //Selection d'une broche PWM

  float duty_cyle[11] = {0, 0.1, 0.2, 0.3, 0.4, 0.5, 0.6, 0.7, 0.8, 0.9, 1.0};//Création d'un tableau avec les différents rapports cycliques
  
  void setup() {
  
      pinMode(pin_led, OUTPUT);  //Mise en sortie de la broche LED
  
  }//Fin setup
  
  void loop() {
  
      for(int i=0;i<11;i++) 
      {
          int value_r = duty_cyle[i]*255.0; //Conversion d'une valeur entre 0 et 1 en une valeur entre 0 et 255
          analogWrite(pin_led, value_r); //Change le rapport cyclique pendant 3 s
          delay(600);        //Attend 0.6s
      }
      
  
  }//Fin loop

\end{Cpp}

\subsection{Trame de commande servomoteurs}

\begin{graphic}{0.8}{0.4}{0}{62}{-1}{8}{t(ms)}{Tension(V)}{Trame PWM}

    \addPoints{blue}{(0,5)(1,5)(1,0)(20,0)(20,5)(21,5)(21,0)(40,0)(40,5)(41,5)(41,0)(70,0)}
    \addLegend{Ton=1ms (180°)}
    
\end{graphic}


\subsection{Branchement d'un servo-moteur}

\begin{items}{blue}{\Bullet}
  \item Câble noir ou marron : GND 
  \item Câble rouge : +5V
  \item Câble blanc ou jaune : Signal Arduino (11)
\end{items}


\subsection{Code Arduino}


\begin{Cpp}{Variation de la position d'un \servo}

  #include <Servo.h>      //Inclusion de la bibliothèque Servo
  Servo myservo;  // Création d'un objet Servo
  int pos = 0;    //Angle du servomoteur
  
  void setup() {

    myservo.attach(11);  //Choix de la broche du servo moteur

  }
  
  void loop() {

    for (pos = 0; pos <= 180; pos += 1) { //Parcours la plage angulaire [0-180] degré par degré

      myservo.write(pos);              //Actualise la position 
      delay(15);                       //Attend 15 ms avant l'actualisation

    }//Fin for

    for (pos = 180; pos >= 0; pos -= 1) {     //Parcours la plage angulaire [0-180] degré par degré

      myservo.write(pos);              //Actualise la position 
      delay(15);                       //Attend 15 ms avant l'actualisation

    }//Fin for
  }//Fin loop

\end{Cpp}


\section{Caractéristiques}


\subsection{Electriques}


\begin{items}{blue}{\Bullet}
  \item Tension de commande et d'alimentation : \~5V
\end{items}

\subsection{Mécaniques}

\begin{items}{blue}{\Bullet}
  \item Couple de sortie (Nm)
  \item Vitesse de rotation (degré/temps)
\end{items}
