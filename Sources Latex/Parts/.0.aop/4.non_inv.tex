


\chapter{Montage non-inverseur}


\section{Présentation}

Ce montage amplifie la tension $V_e$ par un \bold{gain $A_0$ positif}. \\
L’amplificateur reste en mode linéaire si $Ve < Vcc_{sat} \cdot A_0$

\section{Montage}

\img{\rootImages/non_inv.png}{Le montage amplificateur non inverseur}{0.8}

\section{Démonstration}

Un AOP en mode linéaire impose $\varepsilon=0$
D'où $E_+ =E_-$ 

$$E_+=V_e$$
$$E_-=V_s \cdot \frac{R_2}{R_1+R_2}$$

\begin{align}
E_+ = E_- &\Rightarrow V_e = V_s \cdot \frac{R_2}{R_1+R_2}\\
 &\Rightarrow \frac{V_e}{V_s} = \frac{R_2}{R_1+R_2}\\
 &\Rightarrow V_s = V_e \cdot \frac{R_1+R_2}{R_2} \\
\end{align}

%avec $A_0=\frac{R_1+R_2}{R_2}$

\subsection{Exemple}


\begin{exemple}
On souhaite amplifier un signal sinusoïdal par un coefficient $k=5$.\\
On peut donc utiliser le montage précédent. \\
On prendra $R_1=1 k\Omega$ et $R_2=4k\Omega$ pour avoir $A_0=5$
\img{\rootImages/inv_img.png}{Amplification du signal noir par 5}{0.4}
\end{exemple}


