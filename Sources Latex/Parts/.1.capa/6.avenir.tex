\chapter{Avenir du condensateur}

Nous avons donc vu dans les parties précédentes que le condensateur est un élément essentielles à tout circuits électronique. \\
Nous nous sommes donc posé la question suivante : Le condensateur sera-t-il amené à disparaître dans le futur pour être remplacé par un autre composant ? \\

Il est évident que l'importance du condensateur est telle qu'il parait irremplaçable. \\ 
En effet, on ne peut se passer d'un telle composant dans un circuit électronique.\\
 Cependant, depuis quelques années un composant est utilisé de plus en plus : le super-condensateur. \\

\section{Les supercondensateurs}

Les supercondensateurs sont une sous-catégorie des condensateurs électrolytiques.\\ 
Ils permettent de stocker une très grande quantité d’énergie grâce à une combinaison de 2 technologies de capacité. \\

La capacité double couche que l’on retrouve dans les condensateurs électrolytiques et la pseudo-capacité. \\
L’une est électrostatique et l’autre électrochimique.\\


Cela leur permet de combiner les caractéristiques des condensateurs ordinaires et celles des batteries.\\ 
En effet grâce à ces technologies ils peuvent atteindre des capacités allant jusqu’à 12 000F, tout en ayant des temps de charge et décharge très rapide comparable aux condensateurs ordinaires.\\
 Toutes ces caractéristiques en feraient de bons candidats pour remplacer nos batteries au lithium.\\
 
 Mais c’était sans compter leurs défauts. Ils sont en effet très chers à produire, possèdent une faible énergie spécifique (Rapport en Wh/kg entre l’énergie électrique fournie par unité de temps et la masse du convertisseur) et une tension de décharge linéaire.\\
 Cela entraînerait de grandes chutes de tension d’alimentation très rapidement.\\

Le supercondensateur sera probablement l'avenir pour le stockage d'électricités, qui a terme remplacera sûrement les batteries des véhicules électriques.\\

Cependant, il est trop puissant pour le mettre dans la majorité des circuits électroniques. La science progresse énormément dans ce domaine.
