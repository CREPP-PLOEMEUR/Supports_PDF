\chapter{Langage HTML}

Le HTML \footnote{HyperText Markup Language} est un langage contenant des balises, c'est à dire des 
marqueurs spécifiques pour organiser la page Web.

Il existe deux types principaux de balises: 

\begin{items}{blue}{\Triangle}
\item Les balises en paire (Par exemple $<h1></h1>$)
\item Les balises orphelines (Par exemple $<img>$)
\end{items}

Une balise commence par un chevron ouvrant et se termine par un chevron fermant.\\
Toutes les balises fermantes (pour les balises en paire) sont de la forme $</nom_balise>$

Toute page HTML commencer par la balise \lbl{blue}{html}{html}

\messageBox{Remarque}{green}{white}{Pour mettre du code en commentaire, c'est à dire ne pas le visualiser dans la page Web de rendu, il faut mettre le code entre \lbl{green}{CMT}{$<--!$} et \lbl{green}{CMT}{$-->$}}{white}


\begin{Html}{Première balise HTML}
<html>
    <!-- Ceci est mon début de page HTML --> 
</html>
\end{Html}

\section{La forme de la page}

La page Web est scindée en 2 entitées: 

\begin{items}{blue}{\Triangle}
\item L'en-tête (header), marqué avec la balise \lbl{green}{CMT}{$<head></head>$}
\item Le corps (body), marqué avec la balise \lbl{green}{CMT}{$<body></body>$}
\end{items}

\begin{Html}{Page minimaliste HTML}
<html>
    <head>
        <!-- Ceci est un header --> 
    </head>

    <body>
        <!--! Ceci est un body -->
    </body>
</html>
\end{Html}


\section{L'en-tête}

L'en-tête va contenir les informations et les paramètres de la page, notamment : 

\begin{items}{blue}{\Triangle}
    \item Le titre de la page
    \item Les importations des bibliothèques (feuilles de style)
    \item Les icônes
    \item L'encodage de la page (UTF-8)
    \end{items}


\begin{Html}{L'en-tête}
<head>
        <--! Titre en haut de la page -->
        <title>Titre de la page</title> 

        <--! Ajout d'une feuille de style -->
        <link rel="stylesheet" href="style.css" type="text/css"> 

        <--! Ajout d'une icone -->
        <link rel="icon" href="icone.ico">  
        
        <--! Encodage UTF-8 -->
        <meta charset="utf-8">  

</head>
    \end{Html}



\section{Le corps}

Le corps va contenir l'ensemble des informations affichées sur la page
    
\begin{items}{blue}{\Triangle}
    \item Les titres, sous-titre, sous-sous-titre
    \item Les paragraphes 
    \item Les images
    \item Les liens
    \item Les sections de code
    \item ...
\end{items}
    
    
\subsection{Ajout des titres}

Un titre en HTML est vue avec un niveau hierarchique. 
Un titre de page est au niveau 1, un sous-titre avec un niveau 2, etc...

Pour mettre un titre, il faut donc écrire \lbl{green}{CMT}{$<h1>Titre</h1>$}, un sous-titre 
c'est \\
\lbl{green}{CMT}{$<h2>Sous-titre</h2>$}


\subsection{Ajout des images}

Pour ajouter une image, il faut connaitre son emplacement dans le système (ordinateur) ou bien sur intenet (url).

Il s'agit de la balise orpheline \lbl{green}{CMT}{$<img>$}

\begin{Html}{Ajout d'une image}
<--! Image locale -->
<img src="monImage.png" alt="Impossible de charger l'image">

<--! Image Internet -->
<img url="www.crepp.oreg/image.png" alt="Impossible de charger l'image">
\end{Html}


\subsection{Ajout des liens}

Pour ajouter une image, il faut connaitre son emplacement dans le système (ordinateur) ou bien sur intenet (url).

Il s'agit de la balise orpheline \lbl{green}{CMT}{$<img>$}

\begin{Html}{Ajout d'un lien}
<--! Lien -->
<a href="monChemin" > Texte du lien</a>
\end{Html}
Les liens permettent de pointer sur d'autres pages, que ce soit sur le serveur courant ou bien un autre.




