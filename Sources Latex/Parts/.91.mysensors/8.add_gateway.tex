\chapter{Configuration de Domoticz}

Une fois que la passerelle est fonctionnelle, nous allons configurer Domoticz pour que la plateforme reçoive 
les données en provenance de la passerelle.

\section{Ajout de la passerelle}

Tout d'abord, allez dans la section \bold{Configuration > Matériel}

\img{\rootImages/hardware.png}{Emplacement du matériel}{0.5}

Ensuite, saisissez les informations suivantes :

\img{\rootImages/add_gateway.png}{Paramétrage de la passerelle}{0.5}

Le port série sélectionné sera celui où est raccordé la passerelle en liaison USB.
Il ne faut pas prendre les noms simplifiés des ports USB (\italic{COM\_XXX}) mais le nom le plus complet.\\
Pour plus de simplicité, veuillez déconnectez tous les autres périphériques du Raspberry-Pi\\.


\section{Recherche des capteurs}

Visualisons les données en provenance de la sonde en allant dans \\
\bold{Configuration > Matériel}\\.

L'ensemble de vos dispositif apparaît. En cas de liste trop longue, saisissez \bold{Gateway} dans la barre de recherche.


\img{\rootImages/search.png}{Recherche de la passerelle}{0.5}


\messageBox{Remarque}{orange}{white}{Si le dispositif n'apparaît pas immédiatement, patientez quelques instants.}{black}

\img{\rootImages/gateway.png}{La passerelle est détectée}{0.4}


%\messageBox{Précision}{green}{white}{Dans notre programme \bold{Sonde\_MySensors.ino}, 
%nous avons défini la valeur du noeud à 30.\\On retrouve bien cette valeur dans la colonne \bold{Idx}}{black}

Nous allons ensuite vérifier que les capteurs de la sonde envoient bien les données. Pour cela, cliquez sur \bold{Configuration}.
La page suivante apparaît : 

\img{\rootImages/home.png}{Page de la Gateway}{0.4}

Pour visualisez les valeurs des capteurs, il faut sélectionner la passerelle avec l'ID de la sonde (ici, 30).

\img{\rootImages/nodes.png}{Sélection de la passerelle}{0.5}

En cliquant dessus, on voit que la partie \bold{Enfants} est mise à jour et contient les 3 capteurs avec les ID définis dans le programme de la sonde (31,32 et 33 en ce qui me concerne)

\img{\rootImages/enfants.png}{Visualisation des enfants}{0.4}


\section{Visualisation des données}

Maintenant que nous savons que la sonde envoi les bonnes données, nous allons ajouter les capteurs dans les dispositifs.
Pour cela, allez dans \bold{Configuration > Dispositifs}, les 3 capteurs de la sonde (Tension batterie, humidité et température) apparaissent dans la liste.\\
Si vous ne les trouvez pas, vous pouver nettoyer la page des capteurs en sélectionnant les capteurs non-utilisés et en le mettant à la poubelle.

\img{\rootImages/disp.png}{Visualisation des capteurs}{0.25}

Les capteurs apparaissent sous les 3 noms suivants : 

\img{\rootImages/names.png}{Nom des capteurs}{0.6}

Pour ajouter un dispositif, il suffit de cliquer sur la flèche verte et de choisir le nom du dispositif.

\img{\rootImages/fleche.png}{Ajout des dispositifs}{0.5}

\img{\rootImages/voltage.png}{Ajout des dispositifs - Sélection du nom}{0.5}

Il suffit de cliquer dans le menu \bold{Mesures}

\img{\rootImages/mesure.png}{Mesures}{0.4}

Et apparaît la tension de la batterie.

\img{\rootImages/volt.png}{tension de la batterie}{0.5}

On procède de même pour l'humidité et la température, les dispositifs seront mis dans l'onglet  \bold{Température}.

\img{\rootImages/temp.png}{Mesures de l'humidité et de la température}{0.4}


Pour visualiser les données, il suffit de cliquer sur le bouton \bold{logs}


